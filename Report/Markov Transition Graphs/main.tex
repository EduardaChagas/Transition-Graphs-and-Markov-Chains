\documentclass[10pt]{article}

\usepackage{sbc-template}
\usepackage[utf8]{inputenc}
\usepackage[portuguese]{babel}
\usepackage[binary-units]{siunitx}
\usepackage{lipsum} 
\usepackage{graphicx,url}
\usepackage{booktabs}
\usepackage{amsfonts,amsmath,bm,bbm}
\usepackage{natbib}
\usepackage{esvect}
\usepackage[colorlinks=true, allcolors=blue]{hyperref} 

\sloppy

\title{Estudo de padrões ordinais com grafos de transição e Cadeias de Markov}


\author{
	Eduarda T.\ C.\ Chagas$^a$,
	Heitor S.\ Ramos$^a$,
	Osvaldo A.\ Rosso$^b$,
	Alejandro C.\ Frery$^c$
}

\address{
$^a$Departamento de Ci\^encia da Computa\c c\~ao, Universidade Federal de Minas Gerais, Brazil
\\
$^b$Instituto de F\'isica, Universidade Federal de Alagoas, Brazil
\\
$^c$School of Mathematics and Statistics, Victoria University of Wellington, New Zealand
}


\begin{document}

\maketitle

\section*{Definição do problema} \label{abstract}

Uma das limitações das técnicas de análise de séries temporais baseadas em padrões ordinais é não permitirem fazer previsões (\textit{forecast}) nem simulações.
Este trabalho explora essa possibilidade através de cadeias de Markov construídas por grafos de transição de padrões ordinais das séries temporais empíricas.

Sejam $\bm Z=(Z_1,\dots,Z_n)$ uma série temporal de valores reais,
$\bm \pi^{(D,\tau)} = (\pi^{(D,\tau)}_1,\dots,\pi^{(D,\tau)}_{N-(D-1)\tau})$ a série de padrões a ela associados (calculados com palavras de dimensão $D$ e atraso $\tau$) e $G=(V,E)$ o grafo de transições obtido a partir de $\bm \pi^{(D,\tau)}$.
A proposta consiste em fazer a junção das evidências de $\bm Z$, $\bm \pi^{(D,\tau)}$, e $G$ para fazer simulações e previsões a respeito da série.

\section*{Métodos de Amostragem} \label{sampling}

O primeiro passo de nossa proposta é analisar $\bm Z$ e $\bm \pi^{(D,\tau)}$.
Para cada padrão observado em $\bm \pi^{(D,\tau)}$, coletamos os dados que o originaram em $\bm Z$.
Consideremos, por exemplo, o caso do padrão $\pi^{(3,1)}_1=b_{j}b_{j+1}b_{j+2}$, e suponhamos que ele corresponde a todas as palavras que satisfazem $z_{j}<z_{j+1}<z_{j+2}$.
Assumimos então que cada padrão será formado por elementos amostrados de uma distribuição normal.
Assim, investigamos duas alternativas de amostragem dos dados.
Todas essas palavras serão coletadas e analisadas para obter:
\begin{itemize}
	\item uma estimativa da distribuição três-variada, ou
	\item estimativas do valor central e estimativas de uma medida de dispersão de cada um dos três valores, por exemplo a média e o desvio padrão.
\end{itemize} 
Teremos, assim, associados ao padrão $\pi^{(3,1)}_1$, 
\begin{itemize}
	\item um modelo $\widehat{\mathcal{D}}(\pi^{(3,1)}_1)$, ou
	\item três médias $\widehat\mu_{b_j}, \widehat\mu_{b_{j+1}}, \widehat\mu_{b_{j+2}}$ e três desvios padrão $s_{b_j}, s_{b_{j+1}}, s_{b_{j+2}}$.
\end{itemize} 

\section*{Métodos de Geração das Cadeias de Markov} \label{markov}

O segundo passo consiste em formar a matriz de transições do grafo $G$, digamos $M$ com base nos padrões ordinais da série temporal analisada.
Por construção, a cadeia é irredutível, e basta com que haja uma única transição entre estados iguais para que a cadeia seja aperiódica.
Com estas propriedades, há uma única distribuição de equilíbrio $\Pi$, que é a solução de $\Pi M=\Pi$.

Dada a série temporal $\bm Z=(Z_1,\dots,Z_n)$, associada à sequência $\bm \pi^{(D,\tau)} = (\pi^{(D,\tau)}_1,\dots,\pi^{(D,\tau)}_{N-(D-1)\tau})$ de padrões, simularemos o evento $Z_{n+1}$ com dois elementos:
\begin{itemize}
	\item o padrão $\pi^{(D,\tau)}_1,\dots,\pi^{(D,\tau)}_{N-(D-1)\tau+1}$ que possui probabilidade máxima de ocorrência em $\Pi$ após o último padrão, e
	\item uma observação do modelo $\mathcal D$ correspondente a esse padrão. Note-se que será necessário obter apenas uma amostra da distribuição marginal de $\mathcal D$ dadas as observações já presentes nos últimos estágios da série.
\end{itemize} 

A nossa previsão da observação que sucede $Z_n$ será o estado de equilíbrio mais plausível que segue ao último padrão que inclui $Z_n$.
Para a construção dos grafos de transição, utilizamos o algoritmo tradicional de cálculo das frequências de transições (TG) e a sua versão ponderada pela amplitude (WATG).

\section*{Medidas de similaridade de Séries Temporais} \label{similarity}

Para a análise da eficiência das predições realizadas utilizamos algumas técnicas de similaridade de séries temporais.
A seguir descrevemos como estas se definem, considerando que $\bm X = (x_1, \dots, x_n)$  e $\bm Y = (y_1, \dots, y_n)$ são respectivamente sequências numéricas representando os valores da série temporal analisada e os valores sintéticos gerados por simulações da cadeia de Markov.

\subsection*{Autocorrelation-based Dissimilarity} \label{ACF}

Autocorrelation-based (ACF) calcula a dissimilaridade entre um par de séries temporais baseado na distância euclidiana dos coeficientes de autocorrelação.
Logo, sejam omega (\(\Omega\)) um vetor de pesos uniformes e \(\hat{\rho}_{X}\) and \(\hat{\rho}_{Y}\) os respectivos vetores de coeficientes de autocorrelação, a medida de dissimilaridade é dada como:
\begin{equation*}
    d(\bm X, \bm Y) = {\{ ( \hat{\rho}_{X} - \hat{\rho}_{Y})^t \bm{\Omega} (\hat{\rho}_{X} - \hat{\rho}_{Y} ) \}}^\frac{1}{2}
\end{equation*}

\subsection*{Dynamic Time Warping distance} \label{DTW}

O algoritmo Dynamic Time Warping (DTW) calcula o alinhamento não-linear (elástico) ótimo para as duas sequências numéricas, através da soma das distâncias ao longo de suas trajetórias.
Desse modo, conseguimos encontrar padrões entre eventos de diferentes ritmos, produzindo uma medida de similaridade mais intuitiva.

O primeiro passo no algoritmo DTW consiste em construir uma grade $n \times n$ ou matriz de distâncias, onde cada ponto $(i, j)$, corresponde a um alinhamento entre os elementos $x_i$ e $y_j$.
O objetivo do algoritmo é alinhar os elementos de $\bm X$ e $\bm Y$, de modo que a distância entre eles seja minimizada.
Assim, sendo $\delta$ função de distância entre duas séries temporais, podemos definir formalmente o problema como uma minimização sobre caminhos de distorção com base na distância cumulativa para cada caminho,
\begin{equation*}
    DTW(\bm X, \bm Y) = \min_w \left[ \sum_{k = 1}^{p} \delta(w_k) \right].
\end{equation*}

Apesar de fornecer um valor relacionado a distância entre duas séries temporais, tal medida não obedece a desigualdade triangular, logo não pode ser considerada uma métrica. 

\subsection*{Temporal Correlation Coefficient} \label{CORT}

A distância de CORT calcula um índex de dissimilaridade entre duas séries temporais de mesmo comprimento que mensura por meio do coeficiente de correlação temporal de primeira ordem a proximidade entre a dinâmica de comportamento das sequências:
Sua fórmula é dada por:
\begin{equation*}
    CORT(\bm X, \bm Y) = \frac{ \sum_{t=1} (x_{t+1} - x_t) ( y_{t+1} - y_t) }{ \sqrt{ \sum_{t=1} (x_{t+1} - x_t)^2} \sqrt{ \sum_{t=1} (y_{t+1} - y_t)^2 }}.
\end{equation*}

\subsection*{Permutation Distribution Distance} \label{BP}

Calcula a similaridade entre duas séries temporais através da distância de suas distribuições de permutação de Bandt-Pompe.
Sejam $\bm P = (p_1, \dots, p_T)$ e $\bm Q = (q_1, \dots, q_T)$ distribuições de permutação de Bandt-Pompe associadas a $\bm X$ e $\bm Y$.
A medida de similaridade é dada pela distância quadrada de Hellinger: 
\begin{equation*}
    D(\bm P, \bm Q) = \frac{1}{\sqrt{2}} \left\|\sqrt{P} - \sqrt{Q} \right\|^2_2.
\end{equation*}
Por satisfazer a desigualdade triangular, ser simétrica, não negativa e limitada entre zero e um, tal medida pode ser considerada uma métrica de similaridade entre séries temporais.

\section*{Resultados} \label{results}

Para avaliar a eficiência de nossa proposta, geramos \num{50} sequências para cada comprimento $N = \{10100, 100100\}$, provenientes de modelos auto-regressivos AR1 definidos da seguinte forma: 
\begin{equation*}
    \bm X_t = \Phi \bm X_{t-1} + \epsilon_t,
\end{equation*}
onde $\Phi = 0.3$ e $\epsilon_t \sim \mathcal{N}(0, 1)$.
Já para o processo de simbolização de Bandt-Pompe e geração dos grafos de transição aplicamos sempre $\mathcal D = 3$ e $\tau = 1$.

Uma vez, geradas as nossas sequências base, escondemos do modelo de cadeias de Markov seus últimos $100$ elementos e realizamos a predição destes.
Para cada predição realizada, realimentamos o grafo de transição de modo que o sistema sempre representasse sua versão mais atual.
Na tabela~\ref{Tab:results}, apresentamos os resultados obtidos após a geração das predições utilizando ambos algoritmos de grafos de transição de padrões ordinais e os métodos de amostragem aqui propostos.
Para avaliar a qualidade da predição, aplicamos quatro medidas de similaridades de séries temporais entre as sequências de $100$ elementos geradas pelo modelo auto-regressivo e suas predições associadas: Autocorrelation-based Dissimilarity (ACF), Dynamic Time Warping (DTW), Temporal Correlation Coefficient (CORT) e o quadrado da distância de Hellinger entre as distribuições de permutação de Bandt-Pompe (BP).

Para validar a diferença estatística entre os resultados obtidos, aplicamos o teste de hipótese t-student. 
Verificamos que não conseguimos rejeitar a hipótese nula dentre os testes realizados e encontrar uma melhor combinação de algoritmos para a análise.
Nesses testes procuramos encontrar diferenças significativas por algoritmo de formação do grafo de transição, tamanho das sequências e métodos de amostragem.

Como podemos observar na tabela~\ref{Tab:results}, com base nos resultados das similaridades obtidas, verificamos que as sequências apresentam:
\begin{itemize}
    \item ACF alto - alta dissimilaridade nos coeficientes de autocorrelação das séries temporais;
    \item DTW alto - grandes distâncias entre as trajetórias;
    \item CORT baixo - as dinâmicas das sequências apresentam uma baixa correlação temporais;
    \item BP baixo - Suas distribuições de permutação são semelhantes.
\end{itemize}
Logo, podemos concluir que embora tais sequências apresentem informações ordinais semelhantes, o método de amostragem não consegue refletir a dinâmica da série temporal original.

\begin{table}[hbt]
	\caption{Resultados das análises de similaridades entre as sequências auto-regressivas e os dados gerados sinteticamente por cadeias de Markov geradas por grafos de transição de padrões ordinais.}
	\label{Tab:results}
	\centering
    \resizebox{15cm}{!}{
    	\begin{tabular}{l|l|l|c|c|c|c}
    		\toprule
    		$n$ & Algoritmo & Amostragem & ACF & DTW & CORT & BP\\ 
    		\midrule 
    		10k & TG & Distribuição três-variada & $0.922 \pm (0.14)$ & $92.113 \pm (6.67)$ & $0.001 \pm (0.118)$ & $0.281 \pm (0.11)$\\
    		\cmidrule(lr){3-7} 
    		& & Medidas estatísticas & $0.914 \pm (0.14)$ & $90.743 \pm (7.01)$ & $0.014 \pm (0.128)$ & $0.373 \pm (0.36)$\\
    		\cmidrule(lr){2-7} 
    		& WATG & Distribuição três-variada & $0.926 \pm (0.14)$ & $91.887 \pm (6.39)$ & $0.022 \pm (0.101)$ & $0.303 \pm (0.21)$\\
    		\cmidrule(lr){3-7} 
    		& & Medidas estatísticas & $0.918 \pm (0.14)$ & $91.995 \pm (7.32)$ & $0.016 \pm (0.116)$ & $0.423 \pm (0.44)$\\
    		\midrule
    		100k & TG & Distribuição três-variada & $0.913 \pm (0.13)$ & $89.835 \pm (5.25)$ & $-0.022 \pm (0.112)$ & $0.299 \pm (0.23)$\\
    		\cmidrule(lr){3-7} 
    		& & Medidas estatísticas & $0.927 \pm (0.13)$ & $90.209 \pm (5.92)$ & $-0.002 \pm (0.114)$ & $0.332 \pm (0.32)$\\
    		\cmidrule(lr){2-7} 
    		& WATG & Distribuição três-variada & $0.943 \pm (0.13)$ & $91.042 \pm (6.22)$ & $-0.039 \pm (0.123)$ & $0.389 \pm (0.43)$\\
    		\cmidrule(lr){3-7} 
    		& & Medidas estatísticas & $0.933 \pm (0.14)$ & $92.316 \pm (5.45)$ & $0.001 \pm (0.112)$ & $0.376 \pm (0.38)$\\
    		\bottomrule
	    \end{tabular}
	 }
\end{table}

\bibliographystyle{agsm}
\bibliography{../../Common/references.bib}

\end{document}